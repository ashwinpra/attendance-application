% Copyright 2014 Jean-Philippe Eisenbarth
%This program is free software: you can 
%redistribute it and/or modify it under the terms of the GNU General Public 
%License as published by the Free Software Foundation, either version 3 of the 
%License, or (at your option) any later version.
%This program is distributed in the hope that it will be useful,but WITHOUT ANY 
%WARRANTY; without even the implied warranty of MERCHANTABILITY or FITNESS FOR A 
%PARTICULAR PURPOSE. See the GNU General Public License for more details.
%You should have received a copy of the GNU General Public License along with 
%this program.  If not, see <http://www.gnu.org/licenses/>.

%Based on the code of Yiannis Lazarides
%http://tex.stackexchange.com/questions/42602/software-requirements-specification-with-latex
%http://tex.stackexchange.com/users/963/yiannis-lazarides
%Also based on the template of Karl E. Wiegers
%http://www.se.rit.edu/~emad/teaching/slides/srs_template_sep14.pdf
%http://karlwiegers.com
\documentclass{scrreprt}
\usepackage{listings}
\usepackage{amsmath}
\usepackage{underscore}
\usepackage{multicol}
\usepackage{tikz}
\usetikzlibrary{positioning}
\usepackage{standalone}
\usepackage[bookmarks=true]{hyperref}
\usepackage[utf8]{inputenc}
\usepackage[english]{babel}
\usepackage{etoolbox}
\makeatletter
\patchcmd{\scr@startchapter}{\if@openright\cleardoublepage\else\clearpage\fi}{}{}{}
\makeatother

\hypersetup{
    bookmarks=false,    % show bookmarks bar?
    pdftitle={Test Plan Document},    % title
    pdfauthor={Jean-Philippe Eisenbarth},                     % author
    pdfsubject={TeX and LaTeX},                        % subject of the document
    pdfkeywords={TeX, LaTeX, graphics, images}, % list of keywords
    colorlinks=true,       % false: boxed links; true: colored links
    linkcolor=blue,       % color of internal links
    citecolor=black,       % color of links to bibliography
    filecolor=black,        % color of file links
    urlcolor=purple,        % color of external links
    linktoc=page            % only page is linked
}
\def\myversion{1.0 }
\date{}%
%\title
\usepackage{hyperref}
\begin{document}

\begin{flushright}
    \rule{16cm}{5pt}\vskip1cm
    \begin{bfseries}
        \Huge{TEST PLAN DOCUMENT}\\
        \vspace{1cm}
        for\\
        \vspace{1cm}
        Attendance Application\\
        \vspace{1cm}
        \LARGE{Version \myversion approved}\\
        \vspace{0cm}
        \begin{align*}
        \text{Prepared by: } \;\;\;\; 
         &\text{Sanskar Mittal} &\text{21CS10057}\\
         &\text{Yash Sirvi} &\text{21CS10083}\\
         &\text{Ashwin Prasanth} &\text{21CS30009}\\
        \end{align*}
        \vspace{1.9cm}
        Indian Institute of Technology Kharagpur\\
        \vspace{1.9cm}
        \today\\
    \end{bfseries}
\end{flushright}

\tableofcontents

\pagebreak

\chapter{Test Plan Identifier}
RS-MTP01.1

\chapter{References}
None Identified.

\chapter{Introduction}
This Master Test Plan is a comprehensive document that outlines the testing approach for the Attendance Application, a software tool specifically designed to track attendance for various types of organizations such as educational institutions, businesses, and other organizations. The primary objective of this software tool is to provide an accurate and efficient attendance tracking system that saves time, reduces errors, and minimizes the need for manual attendance management.\\

To get a more detailed understanding of the Attendance Application's functionality, please refer to the application's Software Requirements Specification (SRS) document.\\

By referring to the SRS document, you can gain a clear understanding of the goals and objectives of the Attendance Application, which will inform the development of the test cases and testing approach. This Master Test Plan will then outline the specific steps and procedures that will be followed to ensure that the software is thoroughly tested and meets the requirements and expectations of the end-users.

\pagebreak

\chapter{Software Risk Issues}
Although our attendance application has measures in place to reduce the chances of fake attendances, there is still a possibility that proxies may occur. This could result in inaccurate attendance records and affect the application's credibility. Therefore, it's important to consider the risk of proxies during testing to ensure the application can detect and prevent them as much as possible. This may involve targeted testing methods and replicating real-world scenarios to mitigate the risk. Any proxy-related issues should be documented and communicated to the development team to resolve them effectively.

\chapter{Features to be Tested}
The following is a list of the areas to be focused on during testing of the application, along with their levels of risk, with H standing for High Risk, M standing for Medium Risk, and L standing for Low Risk 
\begin{itemize}
\item New User Registration (M)
\item Existing User Login (M)
\item Attendance-giving Feature for Student (H)
\item Attendance-taking Feature for Teacher (H)
\item Detection of fake attendances (proxies) (L)
\item Viewing of Detailed Attendance Report for Teacher and Student (L)
\item Notification Alert on Low Attendance for Teacher and Student (L)
\item Ability of Admin to modify Course Details (L)
\item Importing/Exporting Attendance Data (L)
\end{itemize}

\pagebreak

\chapter{Approach}

\section{Testing Levels}
The testing for the Attendance will consist of Unit, System/Integration (combined) and Acceptance test levels. It is hoped that there will be at least one full time independent test person for system/integration testing. However, with the budget constraints and time line established; most testing will be done by the test manager with the development teams participation.

\begin{itemize}
    \item \textbf{UNIT} Testing will be done by the developers and will be approved by them as a team. All unit test information will also be provided to the tester.
    \item \textbf{SYSTEM/INTEGRATION} Testing will be performed by the test manager and development team with assistance. No specific test tools are required for this project. Programs will enter into System/Integration test after all critical defects have been corrected. A program may have up to two Major defects as long as they do not impede testing of the program (I.E. there is a work around for the error).
    \item \textbf{ACCEPTANCE} Testing will be performed by the actual end users with the assistance of the test manager and developers. They will be informed of the acceptance test cases beforehand. Programs will enter into Acceptance test after all critical and major defects have been corrected. A program may have one major defect as long as it does not impede testing of the program (I.E. there is a work around for the error). Prior to final completion of acceptance testing all open critical and major defects MUST be corrected.
\end{itemize}
\section{Defect Reporting}
During the Unit Testing phase, the development team would be responsible for identifying and reporting any defects in the codebase. The defects would be thoroughly documented and reported, including their origin, severity, proposed solutions, and the estimated time needed to resolve the issue. This information would be shared with the testing team and the project manager for further analysis and resolution.\\
In subsequent testing phases, such as System/Integration and Acceptance testing, the focus would be on identifying any defects that were not detected during Unit Testing or that originated from the interactions between different modules or components of the software. In these cases, special attention would be given to defects that should have been caught during Unit Testing but were not, as these defects could potentially be more severe or difficult to resolve at higher testing levels.\\ 
Overall, the goal of the testing process is to ensure that the application meets the required quality standards and is free from any critical defects that could impact the user experience or the integrity of the attendance data.

\chapter{Item Pass/Fail Criteria}
The test process will be completed once the initial set of data has been collected of the students and teachers enrolled in the institution . The administration staff is responsible for ensuring that every piece of data about students and teachers is correct and updated. Once this is done, the application is considered live. \\\\
New data must be collected every semester, while current data must be updated based on requests. Once a new batch of data is collected and verified, it will also be added to the database.

\chapter{Suspension Criteria and Resumption Requirements}
\section{Non-responsive email}
If the email sent after registration is not received, the developer team will work to rectify it. In the meantime, the interface can be checked
\section{Failure of proxy detection}
If the proxy detection facility is not working, the proxy prevention can be tested or even be skipped, depending on the testing team

\chapter{Test Deliverables}
\begin{itemize}
    \item Acceptance test plan
    \item System/Integration test plan
    \item Unit test plans
    \item Test cases document
\end{itemize}

\chapter{Environmental Needs}
The following elements are required to support the overall testing effort at all levels within the application:
\begin{itemize}
    \item Access to the attendance database
    \item Access to the actual attendance (to gauge the effectiveness of proxy detection)
    \item Access to testing data set 
    \item Access to the nightly backup/recovery process
\end{itemize}

\chapter{Staff and Training Needs}
It is preferred that there will be at least one (1) full time tester assigned to the project for the system/integration and acceptance testing phases of the project. This will require assignment of a person at the beginning of the project to participate in reviews, etc. If a separate test person is not available, the project manager/test manager will assume this role.
In order to provide complete and proper testing, the following areas need to be addressed in terms of training:
\begin{itemize}
    \item The developers and testers(s) will need to be trained on the basic understanding of OOPS along with knowledge of the software used such as React and MongoDB
    \item The staff administration will require training on handling csv files which will have student and teachers’ data
    \item The admin must be trained on handling the student data and the application’s admin interface
\end{itemize}

\chapter{Responsibilities}
\begin{center}
\begin{tabular}{|c | c | c | c|} 
 \hline
   & Dev Team & Test Team & Client \\ [0.5ex] 
 \hline
 Acceptance test Documentation \& Execution  &  & x & x \\ 
 \hline
 System/Integration test Documentation \& Execution & x & x &  \\
 \hline
 Unit test Documentation \& Execution & x & x &  \\
 \hline
 System Design Reviews & x & x & x \\
 \hline
 Test Procedures \& Rules & x & x &  \\ [1ex] 
 \hline
\end{tabular}
\end{center}

\chapter{Schedule}
The testing phase of the Attendance Application will be divided into three main parts: Unit Testing, System/Integration Testing, and Acceptance Testing. \\

\begin{itemize}
    \item \textbf{Unit Testing} will take place concurrently in the development phase, and final unit testing will be conducted on 6th April 2023. During this phase, the individual components or modules of the application will be tested independently to ensure that they work as expected.
    \item  \textbf{System/Integration Testing}  will be done on 7th April. This phase will involve testing the application as a whole, after all the individual components have been integrated. The goal is to ensure that the different parts of the application work together seamlessly.
    \item \textbf{Acceptance Testing} will also be done on 7th April. This phase involves testing the application from the perspective of the end-users, to ensure that it meets their requirements and expectations. This testing will be carried out with a group of people under the supervision of the development team, and the results will be evaluated against the acceptance criteria set out in the test plan
\end{itemize}

The result of each testing phase will be analysed as and when it is completed, and the required changes will be implemented in the codebase of the application. \\

Please note that the dates provided are tentative and may be adjusted as necessary, based on the progress of development, availability of resources, and other factors that may impact the testing schedule.

\chapter{Planning Risks and Contingencies}
\section{Delay in Data Collection}
If data is not collected or verified within the given timeframe, it will result in a delay in updating the application. During this period, manual attendance will be necessary and will need to be uploaded to the database after verification.
If the cause of the delay is due to understaffing, current staff members may be required to work overtime as hiring and training temporary staff is not feasible within the given time frame.

\end{document}
